\subsection{Denotations}

Generally I avoid any "bad conventions" that I consider the mathematician use for their narcissistic ego and simple laziness, assuming everyone else suppose them to be quick and convenient, whereas actually they only render vagueness and confusion. Technically, conventions that do not raise confusions, like polymorphisms, are not "bad conventions", and I personally use them a lot. But for the very idea of imposing my (also narcissistic) ideology, I avoid them actively.\\

Lots of mathematicians also love using \(:\) in demonstrating the "type" of a morphism, especially in \(\cat{1}{Set}\), while they even don't use type theory related foundations in their works, but rather the conventional set theory. This note is primarily worked in ZF, and actively notes once uses AoC. Thus I avoid using vague notation of \(:\). \\

Personally I also gives the aesthetics of denotations some priority.\\

I try to include all my notations conventions below.\footnote[1]{When certain notations later in my note are considered to be vague or not single-typed, I may leave a \(^{\ldots}\) to indicate the case, and may fix it later.} They are done recursively. The arguments enclosed by \([\,\cdot\,]\) are in "common conventions", for the convenience of mapping different conventions. \\

\begin{align}
  [\text{f is an object in n-category }\cc] \corres & f\in \cat{0}{\cc} \\
  [\text{f is a morphism in n-category }\cc] \corres & f\in \cat{1}{\cc} \\
  [\text{f is a k-morphism in n-category }\cc (k\leq n)] \corres & f\in \cat{k}{\cc} \\
  [f\in \operatorname{Hom}_{\mathscr{C}}(A,B)] \corres & f\in (\, A \morp{\cc} B\,) \\
  [f: A\to B,\, {A, B}\in \cat{0}{Set}] \corres & f\in (\, A \func B\,) 
\end{align}

These are to say, I only recognise morphisms to be legal elements in a category, in case of unnecessary divisions upon Ob(C) and Hom(C). \\

\begin{align}
  [\text{G and H are isomorphic in groups}] \corres & G\iso{Grp} H \\
  [\text{G and H are the same group}] \corres & G\iso[0]{Grp} H \\
                                                    & G = H
\end{align} 

The equality signs shall read \href{https://ncatlab.org/nlab/show/equivalence}{equivalence}. I am not serious on (7), as in most cases I only use (8), with assumptions working in ZF(C). \\

\begin{align}
  f\in (A\func B) \quad \Longrightarrow \quad & f^{\star}\in (\mathcal{P}(A) \func \mathcal{P}(B)) \\
                                                                           & f^{\star}(S)\,=\, \{x\in B \,|\, (\exists y\in S)(f(y)=x) \}
\end{align} \\

\subsection{Assumptions}

I hate people assuming too much. \\

Again, in this note I mainly work on ZF, optionally with AoC. Thus some claims are natural to be made:
\begin{itemize}
  \item All categories are \textbf{strict} categories.
  \item A
\end{itemize}

\subsection{Lectures}

Eventually, this note is called a "note", because I try to use my own way to collate signifiers flying in Cambridge Mathematics Tripos lectures. You may need \href{https://www.maths.cam.ac.uk/undergrad/files/lecturelist/current/All_Parts.pdf}{this} to look up the exhaustive list of all tripos lectures. \\

Sadly, since I decide to take this note from the beginning of my Part IB auditting II lectures, the notes might not include sufficient IA and IB contents. No physics lecture may exist in my note, as I believe physicians should build their own department.\\

Anywhere in this note, there might spawn out some abbreviations denoting the related lectures, the mappings follow:

\begin{align*}
\intertext{\textbf{Analysis:}}
  \text{Analysis I} &\corres \text{ia-a1} & \text{Analysis and Topology} &\corres \text{ib-a2} \\
  \text{Linear Analysis} &\corres \text{ii-lan} & \text{Complex Analysis} &\corres \text{ib-ca} \\
  \text{Optimisation} &\corres \text{ia-op} & \text{Probability and Measure} &\corres \text{ii-pm}
\intertext{\textbf{Algebra:}}
  \text{Groups} &\corres \text{ia-gp} & \text{Vector and Matrices} &\corres \text{ia-vm} \\
  \text{Linear Algebra} &\corres \text{ib-la} & \text{Groups, Rings and Modules} &\corres \text{ib-grm} \\
  \text{Algebraic Topology} &\corres \text{ii-at} & \text{Galois Theory} &\corres \text{ii-ga} \\
  \text{Commutative Algebra} &\corres \text{iii-com}
\intertext{\textbf{Foundations:}}
  \text{Automata and Formal Languages} &\corres \text{ii-af} & \text{Logic and Set Theory} &\corres \text{ii-ls} \\
  \text{Category Theory} &\corres \text{iii-cat} & \text{Model Theory} &\corres \text{iii-mod}
\intertext{\textbf{Probability:}}
  \text{Probability} &\corres \text{ia-pr} & \text{Markov Chain} &\corres \text{ib-mc} \\
  \text{Probability and Measure} &\corres \text{ii-pm} & \text{Stochastic Finance Models} &\corres \text{ii-sfm} \\
  \text{Statistics} &\corres \text{ib-st} 
\intertext{\textbf{Discrete:}}
  \text{Numbers and Sets} &\corres \text{ia-ns} & \text{Number Theory} &\corres \text{ii-nt} \\
  \text{Graph Theory} &\corres \text{ii-gt} & \text{Number Fields} &\corres \text{ii-nf} \\
  \text{Coding and Cryptography} &\corres \text{ii-cc} \\
\intertext{\textbf{Geometry:}}
  \text{Geometry} &\corres \text{ib-geo} & \text{Riemann Surfaces} &\corres \text{ii-rs} \\
  \text{Algebraic Geometry} &\corres \text{ii-ag} & \text{Algebraic Geometry(III)} &\corres \text{iii-ag}
\intertext{\textbf{Semi-physics:}}
  \text{Vector Calculus} &\corres \text{ia-vc} & \text{Differential Equations} &\corres \text{ia-de} \\
  \text{Methods} &\corres \text{ib-mt} & \text{Variational Principles} &\corres \text{ia-vp}
\end{align*} \\

\subsection{Trivialness}

All mathematics are trivial, up to some extent.